\documentclass[a4paper,10pt]{article}
%\documentclass[a4paper,10pt]{scrartcl}

\usepackage[utf8]{inputenc}

\title{Análisis de datos con Python \\
Ejercicios Clase 1 - UNSAM 2018}
\author{}
\date{}

\pdfinfo{%
  /Title    ()
  /Author   ()
  /Creator  ()
  /Producer ()
  /Subject  ()
  /Keywords ()
}

\begin{document}
\maketitle

\begin{enumerate}
\item Realizar un programa en Python que imprima los cuadrados de todos los numeros naturales entre 1 y 10.
\item Realizar un programa en Python que imprima la suma de todos los numeros impares entre 1 y 10.
\item Realizar un programa que le pida al usuario tres valores a,b,c y tenga como output la cantidad de valores entre a y b que son divisibles por c. \\
Ej: \\
Input: a=10, b=20, c=2\\
Output: 6
\item Realizar un programa en Python que imprima el factorial de un numero dado por el usuario.
\item Ralizar un programa en Python que tenga como input un número y como output imprima unun triángulo rectangulo utilizando el símbolo: *.\\
Ej :\\
Input : 5\\
Output:\\
* \\
** \\
*** \\
**** \\
***** \\

\item  Realizar un progama en Python que le pida al usuario el radio de un círculo y tenga como output el área del mismo.
\item Realizar un progama en Python que convierta grados fahrenheit en grados celcius y de celcius a fahrenheit según las unidades de entrada.\\
Ej:\\
Input: 32F  Output: 0C \\
Input: 0C   Output: 32F
\item Escribir un programa en Python que dados tres valores por el usuario calcule la mediana y el valor medio de los mismos.
\item Si enumeramos todos los numeros naturales menores que 10 que son múltiplos de 3 o de 5 obtenemos 3,5 y 9. La suma de esos multiplos es 23. Hallar la suma de todos los múltiplos de 3 o 5 que sean menores a 1000.


\end{enumerate}

\end{document}

%\item Realizar un programa en Python que cuente la frecuencia de los caracteres de un string. El programa debe pedirle al usuario que ingrese un string y devolver un diccionario.
%\item Realizar un programa que genere un valor aleatorio entre 1 y 9 y el usuario deba adivinarlo. El usuario debe ir probando distintos valores, si no acierta debe volver a probar hasta adivinarlo. Una vez que lo adivine debe imprimirse ``Adivinó!'' y el programa debe finalizar.
%\item Realizar un programa para adivinar un valor entre 1 y 100. Si el numero ingresado es incorrecto y mas pequeño que el numero misterioso programa debe imprimir ``el valor ingresado es mas pequeño'', en el caso que sea mayor debe imprimir ``el valor ingresado es mayor''.
